% Options for packages loaded elsewhere
\PassOptionsToPackage{unicode}{hyperref}
\PassOptionsToPackage{hyphens}{url}
%
\documentclass[
]{article}
\usepackage{amsmath,amssymb}
\usepackage{iftex}
\ifPDFTeX
  \usepackage[T1]{fontenc}
  \usepackage[utf8]{inputenc}
  \usepackage{textcomp} % provide euro and other symbols
\else % if luatex or xetex
  \usepackage{unicode-math} % this also loads fontspec
  \defaultfontfeatures{Scale=MatchLowercase}
  \defaultfontfeatures[\rmfamily]{Ligatures=TeX,Scale=1}
\fi
\usepackage{lmodern}
\ifPDFTeX\else
  % xetex/luatex font selection
\fi
% Use upquote if available, for straight quotes in verbatim environments
\IfFileExists{upquote.sty}{\usepackage{upquote}}{}
\IfFileExists{microtype.sty}{% use microtype if available
  \usepackage[]{microtype}
  \UseMicrotypeSet[protrusion]{basicmath} % disable protrusion for tt fonts
}{}
\makeatletter
\@ifundefined{KOMAClassName}{% if non-KOMA class
  \IfFileExists{parskip.sty}{%
    \usepackage{parskip}
  }{% else
    \setlength{\parindent}{0pt}
    \setlength{\parskip}{6pt plus 2pt minus 1pt}}
}{% if KOMA class
  \KOMAoptions{parskip=half}}
\makeatother
\usepackage{xcolor}
\usepackage[margin=1in]{geometry}
\usepackage{color}
\usepackage{fancyvrb}
\newcommand{\VerbBar}{|}
\newcommand{\VERB}{\Verb[commandchars=\\\{\}]}
\DefineVerbatimEnvironment{Highlighting}{Verbatim}{commandchars=\\\{\}}
% Add ',fontsize=\small' for more characters per line
\usepackage{framed}
\definecolor{shadecolor}{RGB}{248,248,248}
\newenvironment{Shaded}{\begin{snugshade}}{\end{snugshade}}
\newcommand{\AlertTok}[1]{\textcolor[rgb]{0.94,0.16,0.16}{#1}}
\newcommand{\AnnotationTok}[1]{\textcolor[rgb]{0.56,0.35,0.01}{\textbf{\textit{#1}}}}
\newcommand{\AttributeTok}[1]{\textcolor[rgb]{0.13,0.29,0.53}{#1}}
\newcommand{\BaseNTok}[1]{\textcolor[rgb]{0.00,0.00,0.81}{#1}}
\newcommand{\BuiltInTok}[1]{#1}
\newcommand{\CharTok}[1]{\textcolor[rgb]{0.31,0.60,0.02}{#1}}
\newcommand{\CommentTok}[1]{\textcolor[rgb]{0.56,0.35,0.01}{\textit{#1}}}
\newcommand{\CommentVarTok}[1]{\textcolor[rgb]{0.56,0.35,0.01}{\textbf{\textit{#1}}}}
\newcommand{\ConstantTok}[1]{\textcolor[rgb]{0.56,0.35,0.01}{#1}}
\newcommand{\ControlFlowTok}[1]{\textcolor[rgb]{0.13,0.29,0.53}{\textbf{#1}}}
\newcommand{\DataTypeTok}[1]{\textcolor[rgb]{0.13,0.29,0.53}{#1}}
\newcommand{\DecValTok}[1]{\textcolor[rgb]{0.00,0.00,0.81}{#1}}
\newcommand{\DocumentationTok}[1]{\textcolor[rgb]{0.56,0.35,0.01}{\textbf{\textit{#1}}}}
\newcommand{\ErrorTok}[1]{\textcolor[rgb]{0.64,0.00,0.00}{\textbf{#1}}}
\newcommand{\ExtensionTok}[1]{#1}
\newcommand{\FloatTok}[1]{\textcolor[rgb]{0.00,0.00,0.81}{#1}}
\newcommand{\FunctionTok}[1]{\textcolor[rgb]{0.13,0.29,0.53}{\textbf{#1}}}
\newcommand{\ImportTok}[1]{#1}
\newcommand{\InformationTok}[1]{\textcolor[rgb]{0.56,0.35,0.01}{\textbf{\textit{#1}}}}
\newcommand{\KeywordTok}[1]{\textcolor[rgb]{0.13,0.29,0.53}{\textbf{#1}}}
\newcommand{\NormalTok}[1]{#1}
\newcommand{\OperatorTok}[1]{\textcolor[rgb]{0.81,0.36,0.00}{\textbf{#1}}}
\newcommand{\OtherTok}[1]{\textcolor[rgb]{0.56,0.35,0.01}{#1}}
\newcommand{\PreprocessorTok}[1]{\textcolor[rgb]{0.56,0.35,0.01}{\textit{#1}}}
\newcommand{\RegionMarkerTok}[1]{#1}
\newcommand{\SpecialCharTok}[1]{\textcolor[rgb]{0.81,0.36,0.00}{\textbf{#1}}}
\newcommand{\SpecialStringTok}[1]{\textcolor[rgb]{0.31,0.60,0.02}{#1}}
\newcommand{\StringTok}[1]{\textcolor[rgb]{0.31,0.60,0.02}{#1}}
\newcommand{\VariableTok}[1]{\textcolor[rgb]{0.00,0.00,0.00}{#1}}
\newcommand{\VerbatimStringTok}[1]{\textcolor[rgb]{0.31,0.60,0.02}{#1}}
\newcommand{\WarningTok}[1]{\textcolor[rgb]{0.56,0.35,0.01}{\textbf{\textit{#1}}}}
\usepackage{graphicx}
\makeatletter
\def\maxwidth{\ifdim\Gin@nat@width>\linewidth\linewidth\else\Gin@nat@width\fi}
\def\maxheight{\ifdim\Gin@nat@height>\textheight\textheight\else\Gin@nat@height\fi}
\makeatother
% Scale images if necessary, so that they will not overflow the page
% margins by default, and it is still possible to overwrite the defaults
% using explicit options in \includegraphics[width, height, ...]{}
\setkeys{Gin}{width=\maxwidth,height=\maxheight,keepaspectratio}
% Set default figure placement to htbp
\makeatletter
\def\fps@figure{htbp}
\makeatother
\setlength{\emergencystretch}{3em} % prevent overfull lines
\providecommand{\tightlist}{%
  \setlength{\itemsep}{0pt}\setlength{\parskip}{0pt}}
\setcounter{secnumdepth}{5}
\ifLuaTeX
  \usepackage{selnolig}  % disable illegal ligatures
\fi
\IfFileExists{bookmark.sty}{\usepackage{bookmark}}{\usepackage{hyperref}}
\IfFileExists{xurl.sty}{\usepackage{xurl}}{} % add URL line breaks if available
\urlstyle{same}
\hypersetup{
  pdftitle={Projet HAB904B},
  pdfauthor={Basile Pajot (DARWIN), Marion Themeze--Leroy(ECOSYSTEMES),},
  hidelinks,
  pdfcreator={LaTeX via pandoc}}

\title{Projet HAB904B}
\author{Basile Pajot (DARWIN), Marion Themeze--Leroy(ECOSYSTEMES),}
\date{2023-12-21}

\begin{document}
\maketitle

{
\setcounter{tocdepth}{2}
\tableofcontents
}
\hypertarget{lecture-et-exploration-des-donnuxe9es}{%
\section{\texorpdfstring{\textbf{Lecture et exploration des
données}}{Lecture et exploration des données}}\label{lecture-et-exploration-des-donnuxe9es}}

\hypertarget{la-variable-uxe0-expliquer}{%
\subsection{La variable à expliquer}\label{la-variable-uxe0-expliquer}}

La variable d'intérêt à expliquer, est \texttt{Shells}, soit le nombre
de carapaces de tortues récentes trouvées lors des relevés sur le
terrain. Cette variable est un proxy pour estimer le nombre de tortues
mortes d'une année sur l'autre.

\hypertarget{les-variables-explicatives}{%
\subsection{Les variables
explicatives}\label{les-variables-explicatives}}

\begin{itemize}
\item
  \texttt{Prev} est une variable explicative qualitative qui correspond
  à la prévalence pour \emph{Mycoplasma agassizii}, soit le rapport
  entre le nombre de tortues séropositives sur l'effectif total de
  tortues par année pour chaque site.
\item
  \texttt{Site} est une variable qualitative qui correspond au site
  d'échantillonnage. Elle a 10 modalités : le parc national \emph{Big
  Shoals (BS)}, l'aire de gestion de la faune sauvage \emph{Camp
  Blanding (CB)}, l'aire de gestion de la faune sauvage et de
  l'environnement \emph{Cecil Field/Branan Field (CF)}, une
  \emph{propriété privé en Floride centrale (CE)}, le parc national
  \emph{Fort Cooper (FC)}, l'aire de gestion de la faune sauvage
  \emph{Flying Eagle (FE)}, le parc national \emph{Gold Head Branch
  (GH)}, l'aire de gestion de la faune sauvage et de l'environnement
  \emph{Perry Oldenburg (OL)},la station biologique \emph{Ordway-Swisher
  (OR)}, l'aire de gestion de la pêche \emph{Tenoroc Fish (TE)}.
\item
  \texttt{Area} est une variable quantitative qui correspond à l'aire
  couverte par site lors des relevés.
\item
  \texttt{Year} est une variable qualitative qui correspond à l'année
  pour laquelle les relevés ont été faits. Elle a 3 modalités : 2004,
  2005,

  \begin{enumerate}
  \def\labelenumi{\arabic{enumi}.}
  \setcounter{enumi}{2005}
  \tightlist
  \item
  \end{enumerate}
\end{itemize}

\hypertarget{exploration-des-donnuxe9es}{%
\subsection{Exploration des données}\label{exploration-des-donnuxe9es}}

Nous regardons un résumé statistique des variables de notre jeu de
donnée.

\begin{verbatim}
##       Site        year               shells        type         Area      
##  BS     : 3   Length:30          Min.   :0.00   Fresh:30   Min.   : 5.30  
##  CB     : 3   Class :character   1st Qu.:0.00              1st Qu.:15.20  
##  Cent   : 3   Mode  :character   Median :1.00              Median :27.30  
##  CF     : 3                      Mean   :1.80              Mean   :29.02  
##  FC     : 3                      3rd Qu.:2.75              3rd Qu.:43.20  
##  FE     : 3                      Max.   :9.00              Max.   :61.00  
##  (Other):12                                                               
##     density           prev        total_turtle      standprev      
##  Min.   : 1.80   Min.   : 1.00   Min.   : 44.80   Min.   :-0.9163  
##  1st Qu.: 2.50   1st Qu.: 1.20   1st Qu.: 74.75   1st Qu.:-0.9088  
##  Median : 3.50   Median :15.95   Median :104.22   Median :-0.3559  
##  Mean   : 8.76   Mean   :25.45   Mean   :119.72   Mean   : 0.0000  
##  3rd Qu.: 4.80   3rd Qu.:42.05   3rd Qu.:174.90   3rd Qu.: 0.6223  
##  Max.   :33.00   Max.   :80.70   Max.   :200.79   Max.   : 2.0709  
##                                                                    
##        H             Cov_2004         Cov_2005         Cov_2006     
##  Min.   :0.0000   Min.   :0.0000   Min.   :0.0000   Min.   :0.0000  
##  1st Qu.:0.0000   1st Qu.:0.0000   1st Qu.:0.0000   1st Qu.:0.0000  
##  Median :0.0000   Median :0.0000   Median :0.0000   Median :0.0000  
##  Mean   :0.4333   Mean   :0.3333   Mean   :0.3333   Mean   :0.3333  
##  3rd Qu.:1.0000   3rd Qu.:1.0000   3rd Qu.:1.0000   3rd Qu.:1.0000  
##  Max.   :1.0000   Max.   :1.0000   Max.   :1.0000   Max.   :1.0000  
## 
\end{verbatim}

Nous avons un plan d'expérience équilibré avec un même nombre
d'observations par site et par année.\newline 

Pour la prevalence \texttt{prev} la moyenne est supérieure à la médiane,
c'est-à-dire que plus de 50\% des valeurs sont inférieures à la moyenne.
Il en est de même pour le nombre de carapces \texttt{shells}. De plus,
pour la prévalence, la différence entre le troisème quartile et le
minimum est d'environ 40, tout comme la différence entre le maximum et
le 3ème quartile. Ainsi, la gamme de valeurs prise par 25\% des données
est égales à celle prise par 75\% des données. Pour le nombre de
carapaces, la gamme de valeurs prise par 25\% des données est plus de
trois fois supérieur à celle prise par 75\% des données. Ceci est
illustré par les \emph{figures 1 et 2}.

\begin{verbatim}
## NULL
\end{verbatim}

\includegraphics{ProjetHAB940B_2023_PAJOT_THEMEZE--LEROY_files/figure-latex/unnamed-chunk-4-1.pdf}

La distribution du nombre de carapaces ressemble à une distribution de
Poisson.

La figure 3 donne plusieurs informations sur notre jeu de données.

\includegraphics{ProjetHAB940B_2023_PAJOT_THEMEZE--LEROY_files/figure-latex/unnamed-chunk-5-1.pdf}

Tout d'abord, le nombre de carapaces récentes trouvées varie ou non en
fonction des années et cette variation n'est pas la même en fonction des
sites. Le nombre de carapaces récentes par rapport à l'année précédente
reste constante augmente ou diminue. On observe des évolutions
différentes pour les sites : on observe une diminution du nombre de
carapaces pour le site CF sur les trois ans ou un changement de tendance
se traduisant par une diminution puis une augmentation pour le site
CB.\newline Ensuite, la prévalence en fonction des sites peut également
varier en fonction des années. Comme précédemment, cette variation n'est
pas la même en fonction des sites. La prévalence reste constante,
augmente ou diminue. La variation peut être globale sur les trois année
d'étude (augmentation de la prévalence pour le site Old) ou changer
(augmentation puis diminution pour le site Cent). Lorsque le prevalence
augmente d'une année à l'autre
\texttt{prev{[}n{]}\ \textless{}\ prev{[}n+1{]}}, le nombre de carapaces
récentes trouvées l'année suivante augmente
\texttt{shells{[}n+2{]}\textgreater{}shells{[}n+1{]}} (sites CB, Old),
et inversement (sites CF).\newline Ainsi, les variations du nombre de
carapaces récentes trouvées pourrait être expliquée par les variations
de la prévalence.

Il apparaît également que certains sites ont de faibles prévalences (BS,
Ord, FE) quelque soit l'année et d'autres des prévalences élevées (CF,
GH, TE). Ceci concourt avec les observations faites précedemments avec
le résumé de la variable \texttt{prev} et la distribution de la
prévalence. Ainsi, nous pourrons séparer les sites en deux catégories,
ceux à faible ou forte prévalence. Cette variables sera donc traitée de
deux manières : de manière continue et de manière discontinue avec deux
catégories : - faible prévalence (0) : \texttt{Prev} \textless{} 0.25 -
haute prévalence (1) : \texttt{Prev} \textgreater{} 0.25

Nous allons donc essayer de déterminer si la prévalence et l'année
permettent d'expliquer les variations du nombre de carapaces. Nous avons
vu que la prévalence et le nombre de carapaces trouvées diffère entre
les sites et entre les années. Afin de pouvoir nous concentrer sur
l'effet de la prévalence, nous allons mettre un effet aléatoire sur la
variable \texttt{Site}. Nous pourrions faire de même pour la variable
\texttt{Année} mais par souci de simplification, nous allons garder
cette variable en effet fixe.

\includegraphics{ProjetHAB940B_2023_PAJOT_THEMEZE--LEROY_files/figure-latex/unnamed-chunk-6-1.pdf}

D'après la \emph{figure 4} les sites n'ont pas tous la même aire et il
semble qu'un plus grand nombre de carapces sont trouvés sur les sites
avec une plus grande aire. Afin de pourvoir comparer les sites entre
eux, nous allons prendre le rapport entre le nombre de carapaces
trouvées par site et l'aire du site.

D'après les observations faites précedemment, nous souhaitons donc
déterminer si : - le nombre de carapaces récentes trouvées est correlée
avec la prevalence pour \emph{Mycoplasma agassizii} pour une année
donnée. - le nombre de carapces récentes trouvées est plus grand dans
les sites à haute prévalence par rapport aux sites à basse prévalence.

\hypertarget{ajustement-dun-moduxe8le-simple}{%
\section{\texorpdfstring{\textbf{Ajustement d'un modèle
simple}}{Ajustement d'un modèle simple}}\label{ajustement-dun-moduxe8le-simple}}

Nous commençons par un modèle simple \texttt{M1} en considérant
uniquement l'effet de la prévalence sur le nombre de carapces récentes
trouvées.

L'équation (approche fréquentiste) du modèle linéaire simple est la
manière suivante : \newline 
\[\frac {shells} {aire_{site}} = \mu_0 + \beta * prev\] Ceci se traduit
en approche bayesienne par un modèle considérant les hypothèses
suivantes :

\begin{itemize}
\tightlist
\item
  \texttt{shells} suit un loi de poisson de paramètre \(\lambda\)
  (\emph{cf figure 2}), c'est-à-dire que c'est une variable discrète de
  comptage dans un intervale de temps et un espace donnés ; avec une
  variance égale à la moyenne \(E(shells)=V(shells)= \lambda\)
\item
  toutes les observations de \texttt{shells} sont \textbf{indépendantes}
\item
  le logarithme de la moyenne de \texttt{shells} peut être exprimée
  comme la combinaison linéaire des variables explicatives
  sélectionnées.
\item
  que les paramètres à estimer (ordonnée à l'origine et coefficients de
  regression) suivent des lois connues, explicités ci-après.
\end{itemize}

Nous avons donc :
\[Shells_i\:\stackrel{i.i.d}{\sim} \: Pois(\lambda_i)\:avec\:i=1,..30\:le\:nombre\:d'observations\]\\
\[log(\lambda_i) =  log({aire_{i}}) + \mu_0 + \beta * prev_i\] \newline 
Nous utilisons comme priors les distribution suivantes : \newline 
\[\mu_0\:{\sim}\:\mathcal{N}(0, 100)\]
\[\beta\:{\sim}\:\mathcal{N}(0, 100)\]

\begin{verbatim}
## module glm loaded
\end{verbatim}

\begin{verbatim}
## Compiling model graph
##    Resolving undeclared variables
##    Allocating nodes
## Graph information:
##    Observed stochastic nodes: 30
##    Unobserved stochastic nodes: 2
##    Total graph size: 182
## 
## Initializing model
\end{verbatim}

\begin{verbatim}
## Inference for Bugs model at "C:/Users/basil/AppData/Local/Temp/RtmpiuyyIp/model56806cd32283.txt", fit using jags,
##  2 chains, each with 9000 iterations (first 4500 discarded)
##  n.sims = 9000 iterations saved
##          mu.vect sd.vect   2.5%    25%    50%    75%  97.5%  Rhat n.eff
## b.prev     0.562   0.113  0.336  0.488  0.564  0.639  0.780 1.001  9000
## mu.0      -3.000   0.166 -3.331 -3.105 -2.996 -2.891 -2.702 1.001  2600
## deviance  82.173   5.346 80.146 80.648 81.445 82.809 87.288 1.011  9000
## 
## For each parameter, n.eff is a crude measure of effective sample size,
## and Rhat is the potential scale reduction factor (at convergence, Rhat=1).
## 
## DIC info (using the rule, pD = var(deviance)/2)
## pD = 14.3 and DIC = 96.5
## DIC is an estimate of expected predictive error (lower deviance is better).
\end{verbatim}

Nous vérifions que le modèle a bien convergé.

\includegraphics{ProjetHAB940B_2023_PAJOT_THEMEZE--LEROY_files/figure-latex/unnamed-chunk-8-1.pdf}

Les 2 chaînes se mélangent bien et convergent toutes deux. Ceci est
aussi confirmé par la statistique de Gelman-Rubin \(\hat{R}\) qui est
inférieure à 1.1 pour chaque paramètre estimé. Nous notons aussi que
\texttt{n.eff} est supérieur à 100. Nous avons donc des estimations de
nos paramètres qui sont stables et des chaines peu autocorrélées.

\includegraphics{ProjetHAB940B_2023_PAJOT_THEMEZE--LEROY_files/figure-latex/unnamed-chunk-9-1.pdf}

\begin{verbatim}
## [1] 1.811165
\end{verbatim}

Les valeurs estimées du modèle se prettent assez bien à prédire la
distribution du nombre de carapaces. Regardons ce qu'il en est sur le
nombre de carapaces en fonction de la prévalence en fonction du site et
de l'année.

\includegraphics{ProjetHAB940B_2023_PAJOT_THEMEZE--LEROY_files/figure-latex/unnamed-chunk-10-1.pdf}

Notre modèle \texttt{M1} prédit bien la distribution des carapaces mais
elle ne prend pas en compte la variabilité des sites comme le montre les
droites de regressions construites à partir de le moyenne de
distributions postérieures de nos paramètres (\emph{cf.~figure 5}).

\hypertarget{comparaison-de-moduxe8les}{%
\section{\texorpdfstring{\textbf{Comparaison de
modèles}}{Comparaison de modèles}}\label{comparaison-de-moduxe8les}}

En vous aidant de l'article, formez quelques hypothèses et construirez
les modèles correspondant. Ajustez et comparez ces modèles pour
déterminer l'hypothèse la mieux supportée par les données. N'oubliez de
standardiser les variables explicatives continues avec la fonction
scale() de R par exemple. Dans le cas d'une variable année, la
standardisation est un peu différente, n'utilisez pas la fonction
scale(). Si year \textless- c(2006, 2007, \ldots{} 2021) par exemple,
utilisez simplement la variable year - 2005 qui vaut (1, 2, \ldots. 16)
pour avoir toujours une variable entière, mais qui commence à 1 et ne
prend plus de grandes valeurs.

Comme vu dans la partie 1, le nombre de carapace varie en fonction de
l'année et du site. Nous allons donc ajuster différents modèles qui
prennent en compte ces deux variables individuellement ou en les
combinant.

Le modèle null \texttt{M0} :
\[log(\lambda_i) =  log({aire_{i}}) + \mu_0\]

Un modèle avec uniquement l'année \texttt{M3} :
\[log(\lambda_i) =  log({aire_{i}}) + \mu_0 + \alpha_0 * cov_{2004}+\alpha_1 * cov_{2005} +\alpha_2 * cov_{2006}\]

Un modèle avec l'année et la prévalence (continue) \texttt{M4} :
\[log(\lambda_i) =  log({aire_{i}}) + \mu_0 + \alpha_0 * cov_{2004}+\alpha_1 * cov_{2005} +\alpha_2 * cov_{2006} + \beta * prev_i \]

Modèle avec année et site \texttt{M5} :
\[log(\lambda_i) =  log({aire_{i}}) + \mu_0 + \alpha_0 * cov_{2004}+\alpha_1 * cov_{2005} +\alpha_2 * cov_{2006}+ \gamma_j\:\:avec\:\:j=1,...,10\]

Modèle avec année et site et la prévalence\texttt{M6} :
\[log(\lambda_i) =  log({aire_{i}}) + \mu_0 + \alpha_0 * cov_{2004}+\alpha_1 * cov_{2005} +\alpha_2 * cov_{2006}+ \gamma_j + \beta * prev_i\:\:avec\:\:j=1,...,10\]

Modèle avec la prévalence (discontinue) \texttt{M7} :
\[log(\lambda_i) =  log({aire_{i}}) + \mu_0 + \beta * disc\_prev_i\:\:avec\:\:j=1,...,10\]

Modèle avec l'année et la prévalence (discontinue) \texttt{M8} :
\[log(\lambda_i) =  log({aire_{i}}) + \mu_0 + \beta * disc\_prev_i + \alpha_0 * Cov_{2004} + \alpha_1 * Cov_{2005} + \alpha_2*Cov_{2006}\]

Modèle avec l'année, le site et la prévalence (discontinue) \texttt{M9}
:
\[log(\lambda_i) =  log({aire_{i}}) + \mu_0 + \beta * disc\_prev_i + \alpha_0 * Cov_{2004} + \alpha_1 * Cov_{2005} + \alpha_2*Cov_{2006} + \gamma_j \:\: avec j = 1,...,10\]

Partir directement avec les effets aléatoires (prendre en compte la
structuration des données)

\hypertarget{infuxe9rence-et-interpruxe9tation-des-ruxe9sultats}{%
\section{\texorpdfstring{\textbf{Inférence et interprétation des
résultats}}{Inférence et interprétation des résultats}}\label{infuxe9rence-et-interpruxe9tation-des-ruxe9sultats}}

Sur la base du meilleur modèle, donnez les estimations des paramètres
ainsi qu'une mesure de l'incertitude associée. Interprétez vos
résultats.

\hypertarget{discussion}{%
\section{\texorpdfstring{\textbf{Discussion}}{Discussion}}\label{discussion}}

Comparez vos résultats à ceux du papier. Sont-ils semblables ou
différents? Pourquoi selon vous? Si cela vous semble pertinent, proposez
des pistes d'amélioration de l'analyse.

\begin{Shaded}
\begin{Highlighting}[]
\NormalTok{random\_model }\OtherTok{\textless{}{-}} \ControlFlowTok{function}\NormalTok{()\{}
  \CommentTok{\# This model takes into account a random effect for the site}
  \CommentTok{\# Likelihood}
  \ControlFlowTok{for}\NormalTok{(i }\ControlFlowTok{in} \DecValTok{1}\SpecialCharTok{:}\NormalTok{N)\{}
\NormalTok{    S[i] }\SpecialCharTok{\textasciitilde{}} \FunctionTok{dpois}\NormalTok{(lambda[i])}
    \FunctionTok{log}\NormalTok{(lambda[i]) }\OtherTok{\textless{}{-}}\NormalTok{ mu}\FloatTok{.0} \SpecialCharTok{+}\NormalTok{ gamma[site[i]] }\SpecialCharTok{+}\NormalTok{ b.prev }\SpecialCharTok{*}\NormalTok{ prev[i] }\SpecialCharTok{+}\NormalTok{ alpha\_0}\SpecialCharTok{*}\NormalTok{ Cov\_2004[i] }\SpecialCharTok{+}\NormalTok{ alpha\_1 }\SpecialCharTok{*}\NormalTok{ Cov\_2005[i] }\SpecialCharTok{+}\NormalTok{ alpha\_2 }\SpecialCharTok{*}\NormalTok{ Cov\_2006[i] }\SpecialCharTok{+} \FunctionTok{log}\NormalTok{(A[i])}
\NormalTok{  \}}
  \ControlFlowTok{for}\NormalTok{ (j }\ControlFlowTok{in} \DecValTok{1}\SpecialCharTok{:}\NormalTok{nb.sites)\{}
\NormalTok{    gamma[j] }\SpecialCharTok{\textasciitilde{}} \FunctionTok{dnorm}\NormalTok{(}\DecValTok{0}\NormalTok{, tau.s)}
\NormalTok{  \}}
  \CommentTok{\# Priors}
\NormalTok{  mu}\FloatTok{.0} \SpecialCharTok{\textasciitilde{}} \FunctionTok{dnorm}\NormalTok{(}\DecValTok{0}\NormalTok{, }\FloatTok{0.001}\NormalTok{)}
\NormalTok{  sd.s }\SpecialCharTok{\textasciitilde{}} \FunctionTok{dunif}\NormalTok{(}\DecValTok{0}\NormalTok{, }\DecValTok{100}\NormalTok{)}
\NormalTok{  tau.s }\OtherTok{\textless{}{-}} \DecValTok{1} \SpecialCharTok{/}\NormalTok{ (sd.s }\SpecialCharTok{*}\NormalTok{ sd.s)}
\NormalTok{  b.prev }\SpecialCharTok{\textasciitilde{}} \FunctionTok{dnorm}\NormalTok{(}\DecValTok{0}\NormalTok{, }\DecValTok{1}\SpecialCharTok{/}\DecValTok{100}\NormalTok{)}
\NormalTok{  alpha\_0 }\SpecialCharTok{\textasciitilde{}} \FunctionTok{dnorm}\NormalTok{(}\DecValTok{0}\NormalTok{, }\DecValTok{1}\SpecialCharTok{/}\DecValTok{100}\NormalTok{)}
\NormalTok{  alpha\_1 }\SpecialCharTok{\textasciitilde{}} \FunctionTok{dnorm}\NormalTok{(}\DecValTok{0}\NormalTok{, }\DecValTok{1}\SpecialCharTok{/}\DecValTok{100}\NormalTok{)}
\NormalTok{  alpha\_2 }\SpecialCharTok{\textasciitilde{}} \FunctionTok{dnorm}\NormalTok{(}\DecValTok{0}\NormalTok{, }\DecValTok{1}\SpecialCharTok{/}\DecValTok{100}\NormalTok{)}
\NormalTok{\} }

\CommentTok{\# Make the data to use in jags}
\NormalTok{datax }\OtherTok{\textless{}{-}} \FunctionTok{list}\NormalTok{(}
  \AttributeTok{N =}\NormalTok{ gopher}\SpecialCharTok{$}\NormalTok{year }\SpecialCharTok{\%\textgreater{}\%} 
    \FunctionTok{length}\NormalTok{(),}
  \AttributeTok{S =}\NormalTok{ gopher}\SpecialCharTok{$}\NormalTok{shells,}
  \AttributeTok{prev =}\NormalTok{ gopher}\SpecialCharTok{$}\NormalTok{standprev,}
  \AttributeTok{A =}\NormalTok{ gopher}\SpecialCharTok{$}\NormalTok{Area,}
  \AttributeTok{site =}\NormalTok{ gopher}\SpecialCharTok{$}\NormalTok{Site }\SpecialCharTok{\%\textgreater{}\%} 
    \FunctionTok{as.numeric}\NormalTok{(),}
  \AttributeTok{nb.sites =}\NormalTok{ gopher}\SpecialCharTok{$}\NormalTok{Site }\SpecialCharTok{\%\textgreater{}\%} 
    \FunctionTok{unique}\NormalTok{() }\SpecialCharTok{\%\textgreater{}\%} 
    \FunctionTok{length}\NormalTok{(),}
  \AttributeTok{Cov\_2004 =} \FunctionTok{ifelse}\NormalTok{(gopher}\SpecialCharTok{$}\NormalTok{year }\SpecialCharTok{==} \DecValTok{2004}\NormalTok{, }\DecValTok{1}\NormalTok{, }\DecValTok{0}\NormalTok{),}
  \AttributeTok{Cov\_2005 =} \FunctionTok{ifelse}\NormalTok{(gopher}\SpecialCharTok{$}\NormalTok{year }\SpecialCharTok{==} \DecValTok{2005}\NormalTok{, }\DecValTok{1}\NormalTok{, }\DecValTok{0}\NormalTok{),}
  \AttributeTok{Cov\_2006 =} \FunctionTok{ifelse}\NormalTok{(gopher}\SpecialCharTok{$}\NormalTok{year }\SpecialCharTok{==} \DecValTok{2006}\NormalTok{, }\DecValTok{1}\NormalTok{, }\DecValTok{0}\NormalTok{)}
\NormalTok{)}

\CommentTok{\# Make a list of parameters to save}
\NormalTok{params }\OtherTok{=} \FunctionTok{c}\NormalTok{(}\StringTok{"mu.0"}\NormalTok{, }\StringTok{"b.prev"}\NormalTok{, }\StringTok{"sd.s"}\NormalTok{,}\StringTok{"alpha\_0"}\NormalTok{, }\StringTok{"alpha\_1"}\NormalTok{, }\StringTok{"alpha\_2"}\NormalTok{)}

\CommentTok{\# Initial conditions}
\NormalTok{init1 }\OtherTok{\textless{}{-}} \FunctionTok{list}\NormalTok{(}
  \StringTok{"mu.0"} \OtherTok{=} \FloatTok{0.5}\NormalTok{,}
  \StringTok{"b.prev"} \OtherTok{=} \FloatTok{0.5}\NormalTok{,}
  \StringTok{"alpha\_0"} \OtherTok{=} \FloatTok{0.5}\NormalTok{,}
  \StringTok{"alpha\_1"} \OtherTok{=} \FloatTok{0.5}\NormalTok{,}
  \StringTok{"alpha\_2"} \OtherTok{=} \FloatTok{0.5}\NormalTok{,}
  \StringTok{"sd.s"} \OtherTok{=} \FloatTok{0.5}
\NormalTok{)}
\NormalTok{init2 }\OtherTok{\textless{}{-}} \FunctionTok{list}\NormalTok{(}
  \StringTok{"mu.0"} \OtherTok{=} \SpecialCharTok{{-}} \FloatTok{0.5}\NormalTok{,}
  \StringTok{"b.prev"} \OtherTok{=} \SpecialCharTok{{-}} \FloatTok{0.5}\NormalTok{,}
  \StringTok{"alpha\_0"} \OtherTok{=} \SpecialCharTok{{-}}\FloatTok{0.5}\NormalTok{,}
  \StringTok{"alpha\_1"} \OtherTok{=} \SpecialCharTok{{-}}\FloatTok{0.5}\NormalTok{,}
  \StringTok{"alpha\_2"} \OtherTok{=} \SpecialCharTok{{-}}\FloatTok{0.5}\NormalTok{,}
  \StringTok{"sd.s"} \OtherTok{=} \FloatTok{1.5}
\NormalTok{)}
\NormalTok{init }\OtherTok{\textless{}{-}} \FunctionTok{list}\NormalTok{(init1, init2)}

\CommentTok{\# Iteration parameters}
\NormalTok{nb.iterations }\OtherTok{\textless{}{-}} \DecValTok{9000}
\NormalTok{nb.burnin }\OtherTok{\textless{}{-}} \DecValTok{4500}

\CommentTok{\# Run the model}
\NormalTok{M6 }\OtherTok{\textless{}{-}} \FunctionTok{jags}\NormalTok{(}
  \AttributeTok{data =}\NormalTok{ datax,}
  \AttributeTok{parameters.to.save =}\NormalTok{ params,}
  \AttributeTok{inits =}\NormalTok{ init,}
  \AttributeTok{model.file =}\NormalTok{ random\_model,}
  \AttributeTok{n.chains =} \DecValTok{2}\NormalTok{,}
  \AttributeTok{n.iter =}\NormalTok{ nb.iterations,}
  \AttributeTok{n.burnin =}\NormalTok{ nb.burnin,}
  \AttributeTok{n.thin =} \DecValTok{1}
\NormalTok{)}
\end{Highlighting}
\end{Shaded}

\begin{verbatim}
## Compiling model graph
##    Resolving undeclared variables
##    Allocating nodes
## Graph information:
##    Observed stochastic nodes: 30
##    Unobserved stochastic nodes: 16
##    Total graph size: 334
## 
## Initializing model
\end{verbatim}

\begin{Shaded}
\begin{Highlighting}[]
\CommentTok{\# Regardons le modèle et les traces}
\NormalTok{M6}
\end{Highlighting}
\end{Shaded}

\begin{verbatim}
## Inference for Bugs model at "C:/Users/basil/AppData/Local/Temp/RtmpiuyyIp/model568060b24ba0.txt", fit using jags,
##  2 chains, each with 9000 iterations (first 4500 discarded)
##  n.sims = 9000 iterations saved
##          mu.vect sd.vect    2.5%    25%    50%    75%  97.5%  Rhat n.eff
## alpha_0    0.362   5.679 -10.683 -3.474  0.330  4.219 11.483 1.001  6400
## alpha_1   -0.307   5.682 -11.367 -4.132 -0.358  3.531 10.869 1.001  8000
## alpha_2   -0.070   5.677 -11.091 -3.864 -0.120  3.768 11.091 1.001  4900
## b.prev     0.576   0.161   0.278  0.477  0.570  0.668  0.925 1.007   240
## mu.0      -3.077   5.678 -14.194 -6.927 -3.041  0.728  8.023 1.001  6800
## sd.s       0.293   0.278   0.006  0.108  0.230  0.390  0.961 1.100    69
## deviance  79.513   3.381  74.011 77.200 79.053 81.339 87.431 1.006   990
## 
## For each parameter, n.eff is a crude measure of effective sample size,
## and Rhat is the potential scale reduction factor (at convergence, Rhat=1).
## 
## DIC info (using the rule, pD = var(deviance)/2)
## pD = 5.7 and DIC = 85.2
## DIC is an estimate of expected predictive error (lower deviance is better).
\end{verbatim}

\begin{Shaded}
\begin{Highlighting}[]
\FunctionTok{traceplot}\NormalTok{(M6, }\AttributeTok{mfrow=}\FunctionTok{c}\NormalTok{(}\DecValTok{2}\NormalTok{, }\DecValTok{3}\NormalTok{), }\AttributeTok{ask=}\ConstantTok{FALSE}\NormalTok{)}
\end{Highlighting}
\end{Shaded}

\includegraphics{ProjetHAB940B_2023_PAJOT_THEMEZE--LEROY_files/figure-latex/unnamed-chunk-20-1.pdf}

\begin{Shaded}
\begin{Highlighting}[]
\FunctionTok{par}\NormalTok{(}\AttributeTok{mfrow=}\FunctionTok{c}\NormalTok{(}\DecValTok{1}\NormalTok{, }\DecValTok{1}\NormalTok{))}
\end{Highlighting}
\end{Shaded}

\includegraphics{ProjetHAB940B_2023_PAJOT_THEMEZE--LEROY_files/figure-latex/unnamed-chunk-20-2.pdf}

\begin{Shaded}
\begin{Highlighting}[]
\CommentTok{\# On récupère les paramètres et on regarde leur distribution}
\NormalTok{res6 }\OtherTok{\textless{}{-}}\NormalTok{ M6}\SpecialCharTok{$}\NormalTok{BUGSoutput}\SpecialCharTok{$}\NormalTok{sims.matrix }\SpecialCharTok{\%\textgreater{}\%} 
  \FunctionTok{as.data.frame}\NormalTok{()}
\FunctionTok{hist}\NormalTok{(res6}\SpecialCharTok{$}\NormalTok{b.prev)}
\end{Highlighting}
\end{Shaded}

\includegraphics{ProjetHAB940B_2023_PAJOT_THEMEZE--LEROY_files/figure-latex/unnamed-chunk-20-3.pdf}

\begin{Shaded}
\begin{Highlighting}[]
\FunctionTok{hist}\NormalTok{(res6}\SpecialCharTok{$}\NormalTok{mu}\FloatTok{.0}\NormalTok{)}
\end{Highlighting}
\end{Shaded}

\includegraphics{ProjetHAB940B_2023_PAJOT_THEMEZE--LEROY_files/figure-latex/unnamed-chunk-20-4.pdf}

\begin{Shaded}
\begin{Highlighting}[]
\FunctionTok{hist}\NormalTok{(res6}\SpecialCharTok{$}\NormalTok{alpha\_0)}
\end{Highlighting}
\end{Shaded}

\includegraphics{ProjetHAB940B_2023_PAJOT_THEMEZE--LEROY_files/figure-latex/unnamed-chunk-20-5.pdf}

\begin{Shaded}
\begin{Highlighting}[]
\FunctionTok{hist}\NormalTok{(res6}\SpecialCharTok{$}\NormalTok{alpha\_1)}
\end{Highlighting}
\end{Shaded}

\includegraphics{ProjetHAB940B_2023_PAJOT_THEMEZE--LEROY_files/figure-latex/unnamed-chunk-20-6.pdf}

\begin{Shaded}
\begin{Highlighting}[]
\FunctionTok{hist}\NormalTok{(res6}\SpecialCharTok{$}\NormalTok{alpha\_2)}
\end{Highlighting}
\end{Shaded}

\includegraphics{ProjetHAB940B_2023_PAJOT_THEMEZE--LEROY_files/figure-latex/unnamed-chunk-20-7.pdf}

\begin{Shaded}
\begin{Highlighting}[]
\FunctionTok{hist}\NormalTok{(res6}\SpecialCharTok{$}\NormalTok{sd.s)}
\end{Highlighting}
\end{Shaded}

\includegraphics{ProjetHAB940B_2023_PAJOT_THEMEZE--LEROY_files/figure-latex/unnamed-chunk-20-8.pdf}

\begin{Shaded}
\begin{Highlighting}[]
\CommentTok{\# Rétrotransformation}

\NormalTok{shells6 }\OtherTok{\textless{}{-}} \FunctionTok{matrix}\NormalTok{(}\ConstantTok{NA}\NormalTok{,}\AttributeTok{ncol=}\FunctionTok{nrow}\NormalTok{(gopher),}\AttributeTok{nrow=}\FunctionTok{nrow}\NormalTok{(res6))}
\ControlFlowTok{for}\NormalTok{ (i }\ControlFlowTok{in} \DecValTok{1}\SpecialCharTok{:}\FunctionTok{nrow}\NormalTok{(gopher))\{}
\NormalTok{  shells6[,i] }\OtherTok{\textless{}{-}}\NormalTok{gopher}\SpecialCharTok{$}\NormalTok{Area[i] }\SpecialCharTok{*} \FunctionTok{exp}\NormalTok{(res6}\SpecialCharTok{$}\NormalTok{mu}\FloatTok{.0} \SpecialCharTok{+}\NormalTok{ res6}\SpecialCharTok{$}\NormalTok{alpha\_0 }\SpecialCharTok{*}\NormalTok{ gopher}\SpecialCharTok{$}\NormalTok{Cov\_2004[i] }\SpecialCharTok{+}\NormalTok{ res6}\SpecialCharTok{$}\NormalTok{alpha\_1 }\SpecialCharTok{*}\NormalTok{ gopher}\SpecialCharTok{$}\NormalTok{Cov\_2005[i] }\SpecialCharTok{+}\NormalTok{ res6}\SpecialCharTok{$}\NormalTok{alpha\_2 }\SpecialCharTok{*}\NormalTok{ gopher}\SpecialCharTok{$}\NormalTok{Cov\_2006[i] }\SpecialCharTok{+}\NormalTok{ res6}\SpecialCharTok{$}\NormalTok{b.prev }\SpecialCharTok{*}\NormalTok{ gopher}\SpecialCharTok{$}\NormalTok{standprev[i]}\SpecialCharTok{+}\FunctionTok{rnorm}\NormalTok{(}\DecValTok{1}\NormalTok{,}\AttributeTok{mean=}\DecValTok{0}\NormalTok{, }\AttributeTok{sd=}\NormalTok{res6}\SpecialCharTok{$}\NormalTok{sd.s))}
\NormalTok{\}}
\FunctionTok{hist}\NormalTok{(shells6)}
\end{Highlighting}
\end{Shaded}

\includegraphics{ProjetHAB940B_2023_PAJOT_THEMEZE--LEROY_files/figure-latex/unnamed-chunk-20-9.pdf}

\begin{Shaded}
\begin{Highlighting}[]
\FunctionTok{mean}\NormalTok{(shells6)}
\end{Highlighting}
\end{Shaded}

\begin{verbatim}
## [1] 1.671726
\end{verbatim}

\begin{Shaded}
\begin{Highlighting}[]
\CommentTok{\# Récupérons le DIC}
\NormalTok{DIC6 }\OtherTok{\textless{}{-}}\NormalTok{ M6}\SpecialCharTok{$}\NormalTok{BUGSoutput}\SpecialCharTok{$}\NormalTok{DIC}
\end{Highlighting}
\end{Shaded}


\end{document}
